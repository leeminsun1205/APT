\PassOptionsToPackage{table}{xcolor}
\documentclass[12pt]{article}
\usepackage{style}
\usepackage{adjustbox}
% \usepackage[utf8]{inputenc} % Removed for xelatex
% \usepackage[vietnamese]{babel} % Removed for xelatex
% \usepackage[T1]{fontenc} % Removed for xelatex
\usepackage{fontspec}
\usepackage{polyglossia}
\setdefaultlanguage{vietnamese}
\setmainfont{Liberation Serif}

\usepackage{hyperref}
\usepackage{url}
\usepackage{booktabs}
\usepackage{amsfonts}
\usepackage{nicefrac}
\usepackage{microtype}
\usepackage{lipsum}
\usepackage{float}
\usepackage{titlesec}
\usepackage{graphicx}
\usepackage{multicol}
\usepackage{eqparbox}
\usepackage{duckuments}
\usepackage{titletoc}
\usepackage{etoolbox}
\usepackage{enumerate}
\usepackage{doi}
\usepackage{xcolor}
\usepackage{textpos}
\usepackage{algpseudocode}
\usepackage{algorithm}
\usepackage{import}
\usepackage{makeidx}
\usepackage[table]{xcolor}
\usepackage[contents={}]{background}
\usepackage{amsmath,amsthm,amssymb}
\usepackage[shortlabels]{enumitem}
\usepackage{caption}
\usepackage{setspace}
\usepackage{minted}
\usepackage{tkz-tab}
\usepackage{tikz,pgf}
\usepackage{fancyhdr}
\setlength{\headheight}{25pt}
\setlength{\parskip}{0.5em}
\usepackage{multirow}

\usepackage[style=ieee]{biblatex}
\addbibresource{references.bib}
\addto\captionsvietnamese{%
  \renewcommand{\figurename}{Hình}%
  \renewcommand{\tablename}{Bảng}%
}

% Spacing
% --- UI Improvements ---
\definecolor{primaryColor}{RGB}{0, 51, 153} % Deep Blue
\definecolor{secondaryColor}{RGB}{204, 0, 0} % Dark Red

\titleformat{\section}
{\LARGE\bfseries\normalfont\sffamily\color{primaryColor}\raggedright}
  {\color{secondaryColor}\thesection.}
  {1em}
  {\color{primaryColor}\bfseries}
  
\titleformat{\subsection}
{\Large\bfseries\normalfont\sffamily\color{primaryColor}\raggedright}
  {\color{secondaryColor}\thesubsection.}
  {1em}
  {\color{primaryColor}\bfseries}
  
\titleformat{\subsubsection}
{\large\bfseries\normalfont\sffamily\color{primaryColor}\raggedright}
  {\color{secondaryColor}\thesubsubsection.}
  {1em}
  {\color{primaryColor}\bfseries}

% Fancy Header Styling
\pagestyle{fancy}
\fancyhf{}
\lhead{\color{primaryColor}\footnotesize\sffamily \leftmark} 
\rhead{\color{primaryColor}\footnotesize\sffamily CS105.P22 - UIT}
\cfoot{\thepage}
\renewcommand{\headrulewidth}{1pt}
\renewcommand{\headrule}{\hbox to\headwidth{\color{primaryColor}\leaders\hrule height \headrulewidth\hfill}}

\titlespacing*{\section}{0pt}{*0.5}{*0.5}
\titlespacing*{\subsection}{0pt}{*0.5}{*0.5}

\usetikzlibrary{shapes.geometric, arrows.meta, positioning}

\tikzstyle{startstop} = [rectangle, rounded corners, minimum width=4cm, minimum height=1.5cm, text centered, text width=7cm, draw=black, fill=green!30]
\tikzstyle{process} = [rectangle, rounded corners, minimum width=4cm, minimum height=1.5cm, text centered, text width=5cm, draw=black, fill=green!30]
\tikzstyle{arrow} = [thick,->,>=stealth]

\makeindex
\rhead{Place holder}

\numberwithin{equation}{section}
\setcounter{secnumdepth}{4}
\setcounter{tocdepth}{4}
\renewcommand{\contentsname}{Mục Lục}

\begin{document}
\onehalfspacing

% --- Title Page ---
\begin{titlepage}
  \import{./}{cover.tex}
\end{titlepage}
\clearpage
\newpage

\setcounter{page}{1}
\tableofcontents
\clearpage
\newpage

% --- Chapter 1: Introduction ---
\section{Mở đầu (Introduction)}

\subsection{Đặt vấn đề}
\begin{itemize}
  \item Sự bùng nổ của các mô hình đa phương thức lớn (VLMs) như CLIP.
  \item Vấn đề bảo mật: Tấn công đối kháng (Adversarial Attacks) rất dễ đánh lừa mô hình.
  \item \textit{Ví dụ: Minh họa một ví dụ tấn công đối kháng cơ bản.}
\end{itemize}

\subsection{Lý do chọn đề tài}
\begin{itemize}
  \item Các phương pháp phòng thủ hiện tại (như Adversarial Training) quá nặng nề về chi phí tính toán.
  \item Không phù hợp để tinh chỉnh (fine-tune) các mô hình khổng lồ trên tài nguyên hạn chế.
  \item Cần giải pháp nhẹ hơn $\rightarrow$ Prompt Tuning.
\end{itemize}

\subsection{Mục tiêu \& Phạm vi}
\subsubsection{Mục tiêu}
\begin{itemize}
  \item Tăng cường tính bền vững (Robustness) của mô hình trước các tấn công đối kháng.
  \item Giữ nguyên trọng số (weights) của mô hình gốc (frozen backbone).
\end{itemize}
\subsubsection{Phạm vi}
\begin{itemize}
  \item Tập trung vào mô hình \textbf{CLIP (ViT-B/32)}.
  \item Sử dụng các bộ dữ liệu độ phân giải trung bình/thấp như \textbf{CIFAR, TinyImageNet} để phù hợp với tài nguyên thực tế.
\end{itemize}

\subsection{Đóng góp của đề tài}
\begin{itemize}
  \item Triển khai phương pháp APT (Adversarial Prompt Tuning).
  \item Đánh giá trên benchmark rộng (12 bộ dữ liệu).
  \item Phân tích chi phí tính toán so với các phương pháp khác.
\end{itemize}

\clearpage

% --- Chapter 2: Background ---
\section{Cơ sở lý thuyết \& Tổng quan (Background)}

\subsection{Mô hình CLIP}
\begin{itemize}
  \item Kiến trúc Image Encoder \& Text Encoder.
  \item Cơ chế dự đoán Zero-shot (Zero-shot Prediction).
  \item \textit{Hình ảnh: Sơ đồ kiến trúc CLIP.}
\end{itemize}

\subsection{Tấn công đối kháng (Adversarial Attacks)}
\begin{itemize}
  \item Định nghĩa tấn công PGD (Projected Gradient Descent).
  \item Công thức toán học tạo nhiễu $\delta$.
  \item Khái niệm $\epsilon$-ball (giới hạn nhiễu cho phép).
\end{itemize}

\subsection{Prompt Learning}
\begin{itemize}
  \item Khác biệt giữa Hard Prompt (văn bản tự nhiên) và Soft Prompt (vector học được).
\end{itemize}

\subsection{Các công trình liên quan}
\begin{itemize}
  \item Tóm tắt Adversarial Visual Prompting (AVP).
  \item Tóm tắt TeCoA.
  \item \textit{Trích dẫn bài báo gốc ở đây.}
\end{itemize}

\clearpage

% --- Chapter 3: Methodology ---
\section{Phương pháp nghiên cứu (Methodology)}

\subsection{Kiến trúc tổng thể}
\begin{itemize}
  \item \textit{Hình ảnh: Sơ đồ pipeline của APT (vẽ lại từ Hình 2 bài báo gốc).}
  \item Giải thích luồng dữ liệu: Ảnh $\rightarrow$ Image Encoder $\rightarrow$ Similarity $\leftarrow$ Text Encoder $\leftarrow$ Prompt.
\end{itemize}

\subsection{Biểu diễn Prompt (Prompt Parameterization)}
\begin{itemize}
  \item Công thức biểu diễn: $t = [V]_1 [V]_2 ... [V]_M [CLASS]$.
  \item So sánh Unified Context (UC) vs Class-Specific Context (CSC).
  \item \textit{Lý do chọn UC: Tránh overfitting.}
\end{itemize}

\subsection{Chiến lược tối ưu hóa (Optimization Strategy)}
\subsubsection{Hàm mất mát (Loss Function)}
\begin{itemize}
  \item Sử dụng Cross-Entropy Loss trên Logits.
\end{itemize}
\subsubsection{Thuật toán huấn luyện}
\begin{itemize}
  \item \textit{Mã giả (Pseudo-code): Algorithm 2 (On-the-fly).}
  \item Giải thích lý do chọn "On-the-fly" (tạo nhiễu trong quá trình huấn luyện): Cân bằng tốc độ và hiệu quả.
\end{itemize}

\subsection{Phân tích độ phức tạp}
\begin{itemize}
  \item So sánh số lượng tham số cần huấn luyện (vài chục vector prompt) so với Full Fine-tuning (hàng triệu tham số).
\end{itemize}

\clearpage

% --- Chapter 4: Experiments ---
\section{Thực nghiệm \& Kết quả (Experiments)}

\subsection{Cấu hình thực nghiệm}
\begin{itemize}
  \item \textbf{Dataset:} Liệt kê 12 bộ dữ liệu (CIFAR-10, CIFAR-100, TinyImageNet, SUN397-Subset, ...).
  \item \textit{Lý do dùng Subset cho SUN397: Hạn chế tài nguyên.}
  \item \textbf{Metric:} Clean Accuracy, Robust Accuracy (PGD $\epsilon=4/255$), AutoAttack (nếu có).
  \item \textbf{Hardware:} Kaggle P100 / Local iGPU.
\end{itemize}

\subsection{Kết quả chính (Main Results)}
\subsubsection{So sánh APT vs HEP}
\begin{itemize}
  \item \textbf{Bảng 1:} Kết quả trên CIFAR-10, CIFAR-100, TinyImageNet.
  \item \textit{Nhận xét: APT cải thiện Robustness đáng kể so với Baseline.}
\end{itemize}

\subsubsection{Kết quả trên các tập dữ liệu chuyên biệt}
\begin{itemize}
  \item \textbf{Bảng 2:} Kết quả trên EuroSAT, DTD, OxfordPets, ...
  \item \textit{Nhận xét: Hiệu quả trên đa dạng miền dữ liệu.}
\end{itemize}

\subsection{Đánh giá chuyên sâu (Analysis)}
\subsubsection{Tính tổng quát (OOD)}
\begin{itemize}
  \item Kết quả trên \textbf{CIFAR-10-C}.
  \item Đối chiếu với bảng ImageNet OOD từ bài báo gốc.
\end{itemize}

\subsubsection{Hiệu quả dữ liệu (Few-shot)}
\begin{itemize}
  \item So sánh kết quả 1-shot (hoặc 16-shot).
  \item Phân tích khả năng học từ ít dữ liệu.
\end{itemize}

\subsubsection{Chi phí tính toán}
\begin{itemize}
  \item Bảng so sánh Time per Epoch và GFLOPs.
\end{itemize}

\subsection{Thảo luận \& Trực quan hóa}
\subsubsection{Trực quan hóa (Visualization)}
\begin{itemize}
  \item \textit{Hình ảnh: Ảnh gốc vs Ảnh đối kháng (kèm nhãn dự đoán của HEP và APT).}
  \item \textbf{Biểu đồ Robustness Curve:} Accuracy theo mức nhiễu $\epsilon$.
  \item \textbf{Biểu đồ Trade-off:} Clean Acc vs Robust Acc.
\end{itemize}

\subsubsection{Hạn chế}
\begin{itemize}
  \item Chưa chạy trên ImageNet đầy đủ (do tài nguyên).
  \item Chưa thử nghiệm với Backbones khác (như ResNet) hoặc mô hình BLIP/ALIGN (do thiếu Robust Backbone pretrained).
  \item Sử dụng SUN397 Subset thay vì Full set.
\end{itemize}

\clearpage

% --- Chapter 5: Conclusion ---
\section{Kết luận \& Hướng phát triển}

\subsection{Kết luận}
\begin{enumerate}
  \item Đã đề xuất/triển khai giải pháp Prompt Tuning nhẹ cho bài toán Robustness.
  \item Chứng minh hiệu quả trên môi trường tài nguyên hạn chế (Kaggle/Local).
  \item Xác nhận tính đúng đắn trên nhiều bộ dữ liệu (12 datasets).
\end{enumerate}

\subsection{Hạn chế}
\begin{itemize}
  \item Giới hạn về tài nguyên phần cứng dẫn đến việc phải dùng tập con (subset) ở một số dataset.
  \item Chưa mở rộng sang các kiến trúc Transformer khác ngoài ViT-B/32.
\end{itemize}

\subsection{Hướng phát triển}
\begin{itemize}
  \item Nghiên cứu chuyển giao phương pháp sang các mô hình khác (BLIP, ALIGN) khi có điều kiện.
  \item Giải mã ý nghĩa ngữ nghĩa của các vector prompt học được.
\end{itemize}

\clearpage

% --- References ---
\printbibliography

\end{document}