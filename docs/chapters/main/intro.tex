\chapter*{\centering\Large{Mở đầu}}
\addcontentsline{toc}{chapter}{Mở đầu}

\section*{Đặt vấn đề}
Trong những năm gần đây, các mô hình thị giác--ngôn ngữ (Vision--Language Models -- VLMs) phát triển nhanh chóng và đạt hiệu năng cao trên nhiều tác vụ nhận dạng và hiểu nội dung hình ảnh. Một đại diện tiêu biểu là CLIP (Contrastive Language--Image Pre-training) \cite{radford2021learning}, được huấn luyện trên quy mô dữ liệu lớn gồm các cặp ảnh--văn bản thu thập từ Internet, cho phép thực hiện phân loại ảnh theo hướng zero-shot (không cần huấn luyện lại) trên nhiều miền dữ liệu.

Tuy nhiên, song song với hiệu năng trên dữ liệu sạch là những lo ngại về độ tin cậy và an toàn khi triển khai. Nhiều nghiên cứu cho thấy các VLM có thể nhạy cảm trước các tấn công đối kháng (adversarial attacks), trong đó ảnh đầu vào bị nhiễu nhẹ nhưng có thể làm thay đổi dự đoán của mô hình. Điều này đặt ra thách thức đáng kể đối với các kịch bản ứng dụng cần độ tin cậy cao như xe tự lái, hệ thống nhận diện, hay các bài toán liên quan đến y tế và kiểm duyệt nội dung.

\section*{Lý do chọn đề tài}
Một hướng tiếp cận phổ biến để tăng độ bền vững là huấn luyện đối kháng (adversarial training), tức đưa các mẫu đối kháng vào quá trình huấn luyện. Tuy nhiên, khi áp dụng trực tiếp cho các mô hình VLM quy mô lớn như CLIP, cách tiếp cận này thường gặp các hạn chế thực tiễn:

\begin{enumerate}
    \item \textbf{Chi phí tính toán lớn:} Fine-tune toàn bộ mô hình đòi hỏi tài nguyên phần cứng đáng kể và thời gian huấn luyện dài, không phù hợp với nhiều điều kiện triển khai thực tế.
    \item \textbf{Rủi ro quá khớp trong thiết lập ít dữ liệu:} Trong nhiều bài toán downstream, dữ liệu gán nhãn thường hạn chế (few-shot), làm tăng nguy cơ overfitting khi tinh chỉnh nhiều tham số.
\end{enumerate}

Xuất phát từ các hạn chế trên, đề tài lựa chọn hướng tiếp cận \textbf{prompt tuning}, trong đó giữ cố định trọng số mô hình và chỉ tối ưu hoá phần đầu vào văn bản (text prompt). Cụ thể, đồ án tập trung triển khai và đánh giá \textbf{Adversarial Prompt Tuning (APT)} \cite{li2024apt}, một kỹ thuật học prompt dưới huấn luyện đối kháng nhằm cải thiện độ bền vững cho CLIP với số tham số cần học nhỏ.

\section*{Mục tiêu \& Phạm vi}
\subsection*{Mục tiêu}
Mục tiêu của đồ án là triển khai và đánh giá một hướng phòng thủ đối kháng cho CLIP dựa trên prompt tuning. Các mục tiêu cụ thể gồm:
\begin{enumerate}
    \item Hệ thống hoá cơ sở lý thuyết về CLIP, tấn công đối kháng và prompt learning.
    \item Triển khai thuật toán APT để học các vector ngữ cảnh (context vectors) theo bài toán tối ưu hoá cực đại--cực tiểu, trong đó chỉ prompt được cập nhật còn các encoder được giữ cố định.
    \item Đánh giá hiệu quả trên các bộ dữ liệu chuẩn (benchmark), so sánh với các thiết lập baseline dựa trên prompt thủ công theo các độ đo: độ chính xác trên ảnh sạch (clean accuracy) và độ chính xác dưới tấn công đối kháng (robust accuracy) trong cùng threat model.
    \item Phân tích các khía cạnh liên quan đến khả năng khái quát (generalization), bao gồm rủi ro overfitting trong few-shot, đánh giá ngoài phân phối (OOD) khi khả thi, và chi phí tính toán (computational cost).
\end{enumerate}

\subsection*{Phạm vi nghiên cứu}
\begin{itemize}
    \item \textbf{Mô hình:} Sử dụng \textbf{CLIP ViT-B/32} làm backbone. Trọng số của Image Encoder và Text Encoder được đóng băng (frozen); chỉ các tham số prompt (context vectors) được tối ưu.
    \item \textbf{Tấn công:} Sử dụng PGD (Projected Gradient Descent) dưới chuẩn $L_\infty$ để sinh mẫu đối kháng trong quá trình huấn luyện và đánh giá.
    \item \textbf{Dữ liệu:} Đánh giá trên benchmark gồm 15 bộ dữ liệu theo OODRobustBench \cite{li2024apt}. Do giới hạn tài nguyên, các tập dữ liệu lớn (ví dụ ImageNet, SUN397) có thể được thực hiện trên tập con (subset) hoặc trong thiết lập few-shot (ví dụ 16-shot).
\end{itemize}

\section*{Đóng góp của đề tài}
Các đóng góp chính của đồ án được tóm tắt như sau:
\begin{enumerate}
    \item \textbf{Tổng hợp và hệ thống hoá:} Trình bày có hệ thống các khái niệm nền tảng liên quan đến VLM/CLIP, prompt learning và tấn công đối kháng, làm cơ sở cho việc triển khai và đánh giá.
    \item \textbf{Triển khai và tái hiện APT:} Xây dựng quy trình huấn luyện đối kháng ở mức prompt (frozen encoders, chỉ học context vectors) theo công trình gốc, nhằm tạo tiền đề cho các phân tích thực nghiệm về độ bền vững.
    \item \textbf{Đánh giá và phân tích:} Thực hiện đánh giá trên các bộ dữ liệu benchmark trong cùng threat model; phân tích đánh đổi giữa clean accuracy và robust accuracy, đồng thời thảo luận các yếu tố ảnh hưởng đến khả năng khái quát, đánh giá OOD và chi phí tính toán trong phạm vi tài nguyên cho phép.
\end{enumerate}