\documentclass[12pt,a4paper]{article}
\usepackage[utf8]{vietnam}
\usepackage{mathptmx}
\usepackage[left=3.5cm,right=2cm,top=2cm,bottom=2cm]{geometry}
\usepackage{setspace}
\usepackage{array}

\begin{document}
\setstretch{1.2}

\noindent
\begin{minipage}[t]{0.45\textwidth}
    \centering
    ĐẠI HỌC QUỐC GIA TP. HCM \\
    \textbf{TRƯỜNG ĐẠI HỌC} \\
    \underline{\textbf{CÔNG NGHỆ THÔNG TIN}}
\end{minipage}
\hfill
\begin{minipage}[t]{0.5\textwidth}
    \centering
    \textbf{CỘNG HÒA XÃ HỘI CHỦ NGHĨA VIỆT NAM} \\
    \underline{\textbf{Độc lập - Tự do - Hạnh phúc}} \\
    \textit{TP.HCM, ngày \hspace{0.5cm} tháng \hspace{0.5cm} năm 2018}
\end{minipage}

\vspace{1.5cm}

\begin{center}
    \textbf{\Large THÔNG TIN KẾT QUẢ NGHIÊN CỨU}
\end{center}

\vspace{0.5cm}

\noindent \textbf{1. Thông tin chung:}
\begin{itemize}
    \item[-] Tên đề tài: Tăng cường tính bền vững trước tấn công đối kháng cho các mô hình Thị giác-Ngôn ngữ được tiền huấn luyện thông qua tinh chỉnh gợi ý đối kháng
    \item[-] Mã số:
    \item[-] Chủ nhiệm: Lê Minh Nhựt
    \item[-] Thành viên tham gia: Lê Minh Nhựt, Tăng Nhất
    \item[-] Cơ quan chủ trì: Trường Đại học Công nghệ Thông tin.
    \item[-] Thời gian thực hiện: Năm 2026
\end{itemize}

\noindent \textbf{2. Mục tiêu:} Nghiên cứu và nâng cao tính bền vững (robustness) của các mô hình Đa phương thức Ngôn ngữ-Thị giác (Vision-Language Models - VLMs), cụ thể là mô hình CLIP, trước các tấn công đối kháng (adversarial attacks).

\noindent \textbf{3. Tính mới và sáng tạo:} Triển khai phương pháp Adversarial Prompt Tuning (APT) dựa trên quy trình tối ưu hóa tối-đại-thiểu (min-max optimization) và phát hiện "chỉ cần một từ là đủ" (One-word is enough) để tối ưu hóa tài nguyên.

\noindent \textbf{4. Tóm tắt kết quả nghiên cứu:} APT giúp cải thiện tính bền vững đa phương thức (Cross-modal robustness), đạt được sự cân bằng tối ưu giữa độ chính xác trên dữ liệu sạch và dữ liệu đối kháng, đồng thời tiết kiệm tài nguyên huấn luyện.

\noindent \textbf{5. Tên sản phẩm:}

\noindent \textbf{6. Hiệu quả, phương thức chuyển giao kết quả nghiên cứu và khả năng áp dụng:}

\noindent \textbf{7. Hình ảnh, sơ đồ minh họa chính}

\vspace{2cm}

\noindent
\begin{minipage}[t]{0.5\textwidth}
    \centering
    \textbf{Cơ quan Chủ trì} \\
    \textit{(ký, họ và tên, đóng dấu)}
\end{minipage}
\hfill
\begin{minipage}[t]{0.5\textwidth}
    \centering
    \textbf{Chủ nhiệm đề tài} \\
    \textit{(ký, họ và tên)}
\end{minipage}

\end{document}
